% compile in XeLaTeX
\documentclass[12pt, draft]{article}
\usepackage[margin=2.3in]{geometry} % to force more hyphenation

\usepackage{polyglossia}
\setdefaultlanguage{hebrew}
\setotherlanguage{english}
\usepackage{fontspec}
\setmainfont{Frank Ruehl CLM}[HyphenChar=־] % adding this arg in brackets allows e.g. "sof-kol-sof" to have line breaks at the makef

\title{בײַשפּיל: אַריבערטראָגונג}
\author{אײַזיק בלימאַן}
\date{אױגוסט 2019}

\hyphenchar\font=`\־ % make the makef be the default hyphen symbol

\begin{document}

\input{forverts-hyphenation} % the input file produced by running:
% python yiddish_hyphenation_latex.py -i content.txt -o forverts-hyphenation.tex

\maketitle

די רעדאַקטאָרן פֿונעם „װעקער“ האָבן אינעם איצטיקן נומער געמאָלדן, אַז מע עפֿנט אַ קאָנקורס פֿאַר ייִדישע שרײַבערס.

דער זשורנאַל װעט אָננעמען נישט־געדרוקטע דערצײלונגען און אַנדערע קורצע פּראָזעװערק (צװישן 800 און 1,200 װערטער) ביזן 23סטן אױגוסט, װען אַ פּאַנעל פֿון דרײַ אָנגעזעענע ייִדישע שרײַבערס — קטלא קניא, י. פֿלאַװיוס און חײם שטערן — װעט אױסקלײַבן די געװינערס.

די מחברים פֿון די בעסטע װערק װעלן באַקומען געלטפּריזן צװישן 50 און 250 דאָלאַר. לױט דער אָפֿיציעלער מעלדונג װעלן די אַרײַנגעשיקטע װערק „אָפּגעשאַצט װערן דורך אַ רײ פֿאַקטאָרן: אָריגינעלן אינהאַלט, רײַכקײט און שעפֿערישקײט פֿון שפּראַך, שעפֿערישקײט פֿון דער דערצײלונג, גראַמאַטיק און דעם אימפּאַקט אױף דעם לײענער“. מע קען אַרײַנשיקן דערצײלונגען און פּערזענלעכע געשיכטעס – מיט אַן אמתדיקן נאָמען צי אַנאָנימערהײט — אױף דעם אַדרעס: vekerjournal@gmail.com

דער רעדאַקטאָר פֿונעם „װעקער“, באַקאַנט אונטערן פּעננאָמען „ראובֿן“, האָט דערקלערט אַז דער ציל פֿונעם קאָנקורס איז אָנצומוטיקן ייִדיש־רעדערס (מענער און פֿרױען) צו שרײַבן שעפֿעריש. „חסידים האָבן װײניק געלעגנהײטן צו אַרבעטן מיט זײערע טאַלאַנטן“, האָט ראובֿן באַטאָנט, „און אַזעלכע טאַלאַנטן װי שרײַבן, זינגען און מאָלן װערן כּמעט נישט געפֿלעגט [דערמוטיקט] און ליידער צו מאָל אַפֿילו אונטערדריקט“.

אַ צװײטע סיבה פֿאַרן קאָנקורס איז צו דערמוטיקן נײַע שרײַבערס אַרײַנצושיקן זײערע װערק אינעם „װעקער“ גופֿא: „פֿיקציע איז בכלל אַ זשאַנער װאָס מע קען כּמעט נישט בײַ אונדז. מע זעט זײער זעלטן קורצע דערצײלונגען אין חסידישע אױסגאַבעס. אַפֿילו װען שרײַבן איז יאָ געמוטיקט איז עס אױך מיט פֿיל תּנאָים װעלכע נעמען אַרױס דאָס גאַנצע חיות דערפֿון“. די װערק פֿון דרײַ געװינערס, װי אױך די װערק פֿון אַנדערע קאָנקורענטן, װעלן פּובליקירט װערן אין קומעדיקע נומערן „װעקער“ (חודש תּישרי און װײַטער).

דער נײַער שרײַבקאָנקורס בײַם „װעקער“ איז נישט דער אײנציקער אױף דער פֿרומער ייִדישער גאַס. אַ פּריװאַטע ברוקלינער ביבליאָטעק, „Reserve a Book Library“, האָט לעצטנס לאַנצירט אַ שרײַבקאָנקורס, אױך מיט געלטפּריזן. דער קאָנקורס איז אָבער זײער אַנדערש פֿונעם קאָנקורס בײַם „װעקער“: קודם־כּל דאַרפֿן די װערק געשריבן װערן אױף ענגליש פֿאַר אַ יוגנטלעכער לײענערשאַפֿט (צװישן 10 און 18 יאָר). דערצו שטעלט מען זײער אַ סך באַגרענעצונגען אין שײַכות מיטן אינהאַלט: דער העלד פֿון דער דערצײלונג דאַרף בײַקומען עפּעס אַ נסיון, װאָס פֿאַרשטאַרקט זײַן אמונה אינעם אײבערשטן. די דערצײלונג טאָר נישט דערמאָנען קײן קאָנטאַקט צװישן מענער און פֿרױען, טאָר נישט פֿאָרקומען אין ארץ־ישׂראל אָדער מיט העלדן װאָס װילן עולה זײַן, טאָר נישט זאָגן בפֿירוש צי ברמז אַז „עלטערן, רבנים אָדער לערערס זענען נישט גערעכט“. די דערצײלונג טאָר אױך נישט חוזק מאַכן פֿון מיצװת און נישט דערמאָנען קײן מאָדערנע טעכנאָלאָגיעס װאָס ערלעכע ייִדישע קינדער טאָרן נישט ניצן. אַזױ גערופֿענע „אומרעגולערע“ מענטשן, למשל די װאָס גײען אין טעראַפּיע, טאָרן נישט פֿיגורירן אין דער דערצײלונג.

אין קאָנטראַסט מיטן ענגליש־שפּראַכיקן קאָנקורס פֿון דער פּריװאַטער ביבליאָטעק איז דער קאָנקורס פֿון „װעקער“ אַ סך מער אַ פֿרײַע, קינסטלערישע אונטערנעמונג. דער „װעקער“ װיל געבן הדרכה און חיזוק — לאַװ־דװקא צו היטן תּורה־ומצװת (כאָטש דאָס קען נישט שאַטן), נאָר צו אַקטואַליזירן די אינערלעכע טאַלאַנטן װאָס מע האָט אפֿשר נאָך נישט אױסגעפּרוּװט.

אין דעם זינען איז בעסער צו פֿאַרשטײן די כּװנה פֿונעם קאָנקורס פֿון „װעקער“ נישט אין קאָנטעקסט פֿון אַנדערע קאָנקורסן פֿאַר פֿרומע שרײַבערס, נאָר דװקא אין ליכט פֿון די שרײַבקאָנקורסן װאָס זענען פֿאָרגעקומען אין דער װעלטלעכער ייִדישער געשיכטע. אין די 1930ער יאָרן האָט דער ייִװאָ אָרגאַניזירט קאָנקורסן פֿאַר אױטאָביאָגראַפֿיעס פֿון יוגנטלעכע שרײַבערס. פּראָפֿעסאָר צירל קוזניץ, װאָס האָט אָנגעשריבן אַ בוך װעגן דער געשיכטע פֿון ייִװאָ, זאָגט אַז די צילן פֿון די קאָנקורסן זענען געװען סײַ פּראַקטיש, סײַ אידעאָלאָגיש: זײ האָבן נישט נאָר צוגעצױגן אַ נײַעם דור שטיצערס פֿון ייִװאָ און אַרײַנגעבראַכט מאַטעריאַל װאָס האָט געקענט אַרײַנגענומען װערן אין אַרכיװ, נאָר זײ האָבן אױך „געשאַפֿן אַ פֿאָרום פֿאַר דער ייִדישער יוגנט אין די 1930ער זיך אַרײַנצוטראַכטן אין זײערע פּערזענלעכע איבערלעבונגען און אױסצודריקן זײערע אינערלעכע האָפֿענונגען און שעפֿערישע אימפּולסן“. צוליב דעם װאָס יונגע ייִדן האָבן נישט געהאַט די זעלבע פּראָפֿעסיאָנעלע מעגלעכקײטן װי נישט־ייִדן איז זײער באַטײליקונג אין די קאָנקורסן געװען „אַ מין טעראַפּיע, װאָס האָט פֿאַרשטאַרקט זײער חשק בײַצוקומען די אָ שװעריקײטן“ װאָס זײ האָבן גלײַכצײַטיק באַשריבן פֿאַרן אַרכיװ.

אױסצוגן פֿון די אױטאָביאָגראַפֿיעס קען מען געפֿינען אױף ענגליש אינעם בוך „Awakening Lives“, צונױפֿגעשטעלט פֿון יחיאל (דזשעפֿרי) שאַנדלער.

אין 1942 האָט דער ייִװאָ געעפֿנט אַ קאָנקורס פֿאַר אױטאָביאָגראַפֿיעס פֿון ייִדישע אימיגראַנטן כּדי בעסער צו פֿאַרשטײן — און נאָך אַ מאָל, צו פֿאַראײביקן אינעם אַרכיװ — די איבערלעבונגען פֿון יחידים װאָס האָבן פֿאַרלאָזט די אַלטע הײם זוכן אַ נײַ לעבן אין אַמעריקע. טײל פֿון די העכער 200 אױטאָביאָגראַפֿיעס װאָס דער ייִװאָ האָט אָנגעזאַמלט זענען אַרױס אין בוכפֿאָרעם אױף ענגליש אונטערן טיטל „My Future is in America“, רעדאַגירט פֿון דזשאַסעלין כּהן און דניאל סאָיער.

צװישן די נײַסטע פּרוּװן צו זאַמלען ייִדיש־ליטעראַרישן מאַטעריאַל דורך אַ קאָנקורס איז געװען דער קאָנקורס פֿון דער ייִדישיסטישער אָרגאַניזאַציע „יוגנטרוף“ אין 2012. דער ליטעראַרישער פֿאַרנעם איז געװען אַ ביסל ברײטער װי דער איצטיקער קאָנקורס פֿון „װעקער“, װײַל „יוגנטרוף“ האָט אָנגענומען סײַ דערצײלונגען און אַרטיקלען, סײַ פּאָעזיע. װאָס שײַך דער צאָל פּאָטענציעלע קאָנקורענטן איז אָבער דער פֿאַרנעם פֿון „װעקער“ אַװדאי גרעסער. (אַגבֿ, דער אינטערנעץ־פֿאָרום „קאַװע־שטיבל“, װאָס איז אַפֿיליִיִרט מיטן „װעקער“, האָט פֿריִער געפֿירט אַ קאָנקורס פֿאַר פּאָעטן. װעגן דעם איז אַרױס אין „פֿאָרװערטס“ אַן אַרטיקל פֿון ד״ר שלום בערגער.)

מיר ייִדישיסטן, װאָס האַלטן אַזױ פֿון דער שײנער ייִדישער ליטעראַטור פֿון אַ מאָל, װאָלטן באמת געדאַרפֿט אונטערשטיצן אַזעלכע אונטערנעמונגען פֿון חסידישע ייִדן, װעלכע שטעלן זיך פֿאַר אַ ציל צו פֿאַרמערן די צאָל ייִדיש־שרײַבערס און דערהײבן דעם ניװאָ פֿון זײער קונסט. װי אַזױ קען מען שטיצן דעם שרײַבקאָנקורס? אינעם תּישרי־נומער װעט מען אַנאָנסירן די געװינערס און דרוקן זײערע װערק. זאָל מען באַשטעלן און איבערלײענען דעם קומעדיקן נומער, און פֿאַרשפּרײטן די ידיעה אַז שעפֿעריש שרײַבן אױף ייִדיש װעט נאָך האָבן אַ קיום צװישן די ייִדיש־רעדנדיקע מאַסן!

\end{document}
